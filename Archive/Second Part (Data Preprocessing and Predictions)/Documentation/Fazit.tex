\section{Fazit und Ausblick}\label{sec:Fazit und Ausblick}
Die vorliegende Arbeit demonstriert erfolgreich die Anwendung von Machine Learning-Techniken zur Vorhersage von Aktienpreisveränderungen, speziell am Beispiel von Adobe Inc. Durch die systematische Erfassung und Analyse von Daten, die Implementierung verschiedener Machine Learning Modelle und die Optimierung ihrer Hyperparameter konnte eine effektive Methode zur Generierung von Handelssignalen entwickelt werden. Die Modelle wurden sorgfältig ausgewählt und anhand ihrer Performance mittels Metriken wie dem Mean Squared Error und dem R2 Score evaluiert. Die Anwendung von RandomizedSearchCV zur Optimierung der Hyperparameter und Optuna zur Bestimmung optimaler Schwellwerte für Handelssignale hat sich als praktikabel und effizient erwiesen.
\par
Das beste Modell, das identifiziert wurde, ist die Lineare Regression, die einen Gesamtscore von 4.78 erreicht hat. Die festgelegten Schwellenwerte für Kauf- und Verkaufsentscheidungen sind 0,46 \% bzw. -0,011 \%. Durch die Anwendung dieser Schwellenwerte konnte ein Depotwert von 1156,76 € erreicht werden, was eine signifikante Steigerung von 33,93 \% im Vergleich zur Haltstrategie darstellt.
\par
In Zukunft könnte die Forschung in mehrere Richtungen erweitert werden, um die Genauigkeit und Zuverlässigkeit der Vorhersagen weiter zu verbessern. Erstens wäre die Integration zusätzlicher Datenquellen, wie soziale Medien denkbar, um ein umfassenderes Bild des Marktumfelds zu erhalten und die Modelle auf eine breitere Datenbasis zu stützen. Weiterhin wäre die Entwicklung einer dynamischen Anpassung der Modellparameter in Echtzeit eine wertvolle Ergänzung, um schnell auf Marktveränderungen reagieren zu können. Letztendlich könnte die Implementierung und das Backtesting der entwickelten Handelssignale auf historischen Daten dazu beitragen, die praktische Anwendbarkeit und Robustheit der Handelsstrategie unter verschiedenen Marktbedingungen zu testen und zu validieren.