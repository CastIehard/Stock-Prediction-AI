%%%%%%%%%%%%%%%%%%%%%%%%%%%%
%%   Zusaetzliche Pakete  %%
%%%%%%%%%%%%%%%%%%%%%%%%%%%%
\sloppy

\usepackage{enumerate}%Für Aufzählungen
\usepackage{fancyhdr}%Für Kopf- und Fußzeile
\usepackage{a4wide}%Für breitere Seiten
\usepackage{graphicx}%Für Bilder
\usepackage{float}%Für Bilder
\usepackage{subcaption}%Für Bilder
\usepackage{titlesec}%Für Formatierung der Überschriften
\usepackage{enumitem}%Für Aufzählungen
\usepackage{xcolor}%Für Farben
\usepackage{parskip}%Für Abstand zwischen Absätzen

%To include code
\usepackage{listings}
\renewcommand{\lstlistlistingname}{Codeverzeichnis}
\renewcommand{\lstlistingname}{Code} % Optional, ändert die Beschriftung einzelner Listings
\lstset{
    basicstyle=\ttfamily\small\color{white}, % Sets the basic style
    backgroundcolor=\color{black},           % Sets background color to black
    commentstyle=\color{green},              % Comment color
    keywordstyle=\color{cyan},               % Keyword color
    stringstyle=\color{red},                 % String literal color
    showstringspaces=false,                  % Do not show spaces in strings as special character
    numbers=left,                            % Line numbers on the left
    numberstyle=\tiny\color{gray},           % Style of line numbers
    breaklines=true,                         % Automatic line breaking
    captionpos=b,                            % Sets the caption position to bottom
    frame=single                             % Adds a frame around the code
}

\usepackage{colortbl}%Für Farben in Tabellen
\usepackage{caption}%Für Bildunterschriften
\usepackage{nameref}%Für Verweise auf Namen

\usepackage[utf8]{inputenc}%Für Umlaute
\usepackage[ngerman]{babel}%Für deutsche Sprache
\usepackage{lmodern} %Für Schriftart
\usepackage{booktabs} %Für Tabellen
\usepackage{hyperref}%Für Hyperlinks

\usepackage[TS1,T1]{fontenc}%Für Schriftart
\usepackage{fourier, heuristica}%Für Schriftart
\usepackage{array}%Für Tabellen

\newcommand{\foo}{\makebox[0pt]{\textbullet}\hskip-0.5pt\vrule width 1pt\hspace{\labelsep}}

\usepackage{flowchart}%Für Flussdiagramme
\usetikzlibrary{calc}%Für Flussdiagramme
\usepackage{makecell,tabularx} %Für Tabellen
\renewcommand\theadfont{\small}
\newcolumntype{L}{>{\raggedright\arraybackslash}X}
\usepackage{siunitx}
\usepackage{amsmath}
\usepackage{helvet}%Für Arial Font
\renewcommand{\familydefault}{\sfdefault}%Für Arial Font
\usepackage{setspace} %Zeilenabstand
\onehalfspacing{} % Aktiviere den eineinhalbfachen Zeilenabstand
\usepackage[a4paper, left=2.5cm, right=2.5cm, top=2.5cm, bottom=2.5cm]{geometry} % Randabstand 2,5 cm
\usepackage{tcolorbox}
% Definition der Box mit schwarzem Rahmen
\newtcolorbox{blackframebox}{
	colframe=black, % Rahmenfarbe der Box
	boxrule=1pt, % Rahmendicke
	arc=0pt, % Abrundung der Ecken
	boxsep=5pt, % Abstand zwischen Text und Rahmen
	left=5pt, % Linker Randabstand
	right=5pt, % Rechter Randabstand
	top=5pt, % Oberer Randabstand
	bottom=0pt % Unterer Randabstand
}

\usepackage[sorting=nyt, backend=biber, style=authoryear]{biblatex}
\setlength{\bibitemsep}{2\itemsep}%Distance between lines in literature
\usepackage{csquotes}
\usepackage[ddmmyyyy]{datetime}
\usepackage{placeins}
\usepackage{colorprofiles}
\usepackage{algorithm} 
\usepackage[export]{adjustbox}

\addbibresource{Literatur.bib}

%%%%%%%%%%%%%%%%%%%%%%%%%%%%%%
%% Definition der Kopfzeile %%
%%%%%%%%%%%%%%%%%%%%%%%%%%%%%%

\pagestyle{fancy}
\fancyhf{}
\fancyhead[L]{\nouppercase{\leftmark}}
\fancyhead[R]{Studienarbeit, Luca Burghard}
\setlength{\headheight}{15pt}

%%%%%%%%%%%%%%%%%%%%%%%%%%%%%%
%% Definition der Fußzeile %%
%%%%%%%%%%%%%%%%%%%%%%%%%%%%%%

\fancyfoot[C]{Seite \thepage}

%%%%%%%%%%%%%%%%%%%%%%%%%%%%%%%%%%%%%%%%%%%%%%%%%%%%%
%%  Definition des Deckblattes und der Titelseite  %%
%%%%%%%%%%%%%%%%%%%%%%%%%%%%%%%%%%%%%%%%%%%%%%%%%%%%%

\newcommand{\CustomTitle}[0]{
	\thispagestyle{empty}
	\begin{figure}
		\begin{minipage}{0.4\linewidth}
			\includegraphics[height=2.0cm,left]{Bilder/SEWLogo.JPG} 
		\end{minipage} 
		\hfill
		\begin{minipage}{0.4\linewidth}
			\includegraphics[height=2.0cm,right]{Bilder/DHLogo.JPG} 
		\end{minipage}
	\end{figure}
  
  \vspace*{\stretch{1}}
  \begin{center}
    \vspace*{\stretch{0.5}}
    \sffamily\bfseries\Huge Vorhersage von Aktienwerten mit maschinellem Lernen\\

	\vspace*{\stretch{0.5}}
    \sffamily\bfseries\Huge Studienarbeit\\	 

    \vspace*{\stretch{0.5}}
    \sffamily\bfseries\large
	von
    \\
   	Luca Burghard\\
    \vspace*{\stretch{0.5}}
    \sffamily\small
  \end{center}

  \vspace*{\stretch{1}}

   \begin{tabular}{l l}
	Bearbeitungszeitraum: & 25.03.2024 - 07.06.2024\\
	Immatrikulationsnummer: & 9209388\\
	Kurs: & TMT21B2\\
	Studiengang: & Mechatronik\\
	Hochschule: & Duale Hochschule Baden Württemberg Karlsruhe\\ 
	Abgabetermin: & 07.06.2024\\
	Prüfer: & Steffen Quadt\\
    \end{tabular}
 \newpage
}

\titlespacing*{\section}
{0pt}{3.5ex plus 1ex minus .2ex}{.2ex plus .2ex}
\titlespacing*{\subsection}
{0pt}{1.5ex plus 1ex minus .2ex}{.2ex plus .2ex}
\titlespacing*{\subsubsection}
{0pt}{1.5ex plus 1ex minus .2ex}{.2ex plus .2ex}

\setcounter{tocdepth}{2}%Tiefe des Inhaltsverzeichnisses
\setcounter{secnumdepth}{3}%Nummerierung der Überschriften

%%%%%%%%%%%%%%%%%%%%%%%%%%%%
%%  Beginn des Dokuments  %%
%%%%%%%%%%%%%%%%%%%%%%%%%%%%